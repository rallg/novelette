% !TeX TS-program = lualatex
% !TeX encoding = UTF-8
\documentclass[../../../demo.tex]{novelettesubdoc}
\begin{document}

\scene{II. Early History}

Let us drop the Mississippi's physical history, and say a word about its
historical history---so to speak.

The world and the books are so accustomed to use, and over-use, the word
`new' in connection with our country, that we early get and permanently
retain the impression that there is nothing old about it. We do of
course know that there are several comparatively old dates in American
history, but the mere figures convey to our minds no just idea, no
distinct realization, of the stretch of time which they represent.
To say that De Soto, the first European who ever saw the Mississippi
River, saw it in 1542, is a remark which states a fact without
providing its interpretation.

The date 1542, standing by itself, means little or nothing to us; but
when one groups a few neighboring historical dates and facts around it,
he adds perspective and color, and then realizes that this is one of the
American dates which is quite respectable for age.

For instance, when the Mississippi was first seen by a European, less
than a quarter of a century had elapsed since the death of Raphael;
the driving out of the Knights-Hospitallers from Rhodes by
the Turks; and the placarding of the Ninety-Five Propositions.
When De Soto took his glimpse of the river,
Michael Angelo's paint was not yet dry on the Last
Judgment in the Sistine Chapel; Elizabeth of England was
a young girl; \ital{Don Quixote} was not yet written;
Shakespeare was not yet born; and over a hundred long years must
still elapse before Englishmen would hear the name of Oliver Cromwell.

Unquestionably the discovery of the Mississippi is a datable fact which
considerably mellows and modifies the shiny newness of our country, and
gives her a most respectable outside-aspect of rustiness and antiquity.

De Soto merely glimpsed the river, then died and was buried in it by his
priests and soldiers. One would expect the priests and the soldiers
to multiply the river's dimensions by ten---the Spanish custom of the
day---and thus move adventurers to explore it. On the contrary, their
narratives when they reached home, did not excite much curiosity.
The Mississippi was left unvisited by Europeans during a term of years which
seems incredible in our energetic days.

One may `sense' the interval to his mind, after a fashion, by dividing it
up in this way: After De Soto glimpsed the river, nearly
a quarter of a century elapsed; then Shakespeare was born,
lived a trifle more than half a century, and died; and when he had been in
his grave considerably more than half a century, the \ital{second} European
saw the Mississippi. In our day we don't allow a hundred and thirty years to
elapse between glimpses of a marvel. If somebody should discover a creek
in the county next to the one that the North Pole is in, Europe and
America would start fifteen costly expeditions thither: one to explore
the creek, and the other fourteen to hunt for each other.


\end{document}
