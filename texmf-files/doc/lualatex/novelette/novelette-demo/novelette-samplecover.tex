% !TeX TS-program = lualatex
% !TeX encoding = UTF-8
\documentclass[cover]{novelette} % Synced with v0.10.
\title{Sample Cover}
\author{Novelette}
\coverimage{novelette-samplecover-nvtc11721644.jpg}
\covertrim{w=11.47in,h=8.5in}
\booksize{w=5.5in,h=8.5in}
\begin{document}

This is an example of how to produce a PDF/X-1a:2001 conformant file,
from an ordinary image of the correct cover size.

First, you must pre-process the included novelette-samplecover.jpg,
using the cmyk4nvt script. This script is in the novelette-scripts
directory. It is a BASH script. Although it is possible to create
an equivalent Windows *.bat script, the Novelette developer does not
currently have access to Windows. Maybe later. Or, if you have recent
Windows with a Linux subsystem, do it there.

The cmyk4nvt script requires ImageMagick.

To pre-process the image: Copy the entire novelette-scripts directory to
your home directory or desktop. Copy the novelette-samplecover.jpg image
into that directory. Enter that directory, and run:

bash cmyk4nvt novelette-samplecover.jpg

That will take awhile, since there is a lot of processing required.

When it finishes, you will have two new files: one of them will be
name novelette-samplecover-softproof.jpg. This softproof is in sRGB
color space, and will show the approximate printing colors in any
sRGB image viewer. Compared to the original, the colors are slightly dull,
and the dark areas are less dark.

You will also have a very large novelette-samplecover-nvtc11721644.jpg file.
This image is in CMYK color space, with 240 percent ink limit, without
the color profile attached, stripped of private metadata. That is the
conformance standard for PDF/X-1a:2001, as used by print-on-demand.

If the number is not 11721644, no problem: It means that your ImageMagick
is a different version. In that case, at the top of the file you are
now reading, change the coverimage name to the correct value.

The file name of this new file is encoded. Do not change it. When Novelette
processes the document, it reads certain image properties, does a calculation,
and compares the result to the file name. If they do not match, then
the image will be rejected.

Then compile the document:

lualatex novelette-samplecover.tex

The resulting PDF will be PDF/X-1a:2001 with Ouput Intent "US Web Coated SWOP"
which is widely used in the USA. Actually, the output intent will be ignored
by the printer, but it is a PDF/X requirement. Within the PDF metadata,
the entire page area is defined as the BleedBox, and the slightly smaller
TrimBox is defined by the covertrim setting (centered in BleedBox).
With this image size, and document settings, the bleed area is 0.125in
on all four sides of the trimmed cover.

What about this text, that you are reading now? It is ignored. When you
process with [cover] option, most interior-specific settings are ignored,
and any text following begin document is ignored. There is no text or font
in the resulting PDF.

If you remove the [cover] option, then the cover image will be ignored,
and this text will be typeset, using mostly default settings.

\end{document}
