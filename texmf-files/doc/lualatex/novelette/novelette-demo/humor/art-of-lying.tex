% !TeX TS-program = lualatex
% !TeX encoding = UTF-8
\documentclass[../demo.tex]{novelette-subdoc}
\mode{shade} % Allowed in subdoc Preamble.
\begin{document}

\begin{opening}
\name{1}
\null
\footnote{Excerpt from an Essay, 1880.}
\subname{Decay of the Art of Lying}
\end{opening}
% Initial I is a word in itself, here. The following space must be placed
% at the start of whatever follows.
\init{I}[ do not mean to suggest] that the \textit{custom} of lying has
suffered any decay or interruption. The Lie, as a Virtue, as a
Principle, is eternal; the Lie, as a recreation, a solace, a refuge in
time of need, the fourth Grace, the tenth Muse, man's best and surest
friend, is immortal, and cannot perish from the earth.

My complaint simply concerns the decay of the \textit{art} of lying.
Not one of us can contemplate the lumbering and slovenly lying of the
present day without grieving to see a noble art so prostituted.

No fact is more firmly established than that lying is a necessity of our
circumstances. The deduction that it is then a Virtue goes without
saying. No virtue can reach its highest usefulness without careful and
diligent cultivation. Therefore, it goes without saying that this one
ought to be taught in the public schools---even in the newspapers. What
chance has the ignorant uncultivated liar against the educated expert?
What chance have I against a lawyer? \textit{Judicious} lying
is what the world needs. I sometimes think it were even better and safer
not to lie at all than to lie injudiciously. An awkward, unscientific
lie is often as ineffectual as the truth.

Now let us see what philosophers say. Note the venerable proverb:
Children and fools \textit{always} speak the truth. The deduction is plain:
adults and wise persons \textit{never} speak it.

Parkman, the historian,
says, ``The principle of truth may itself be carried into an absurdity.''
In another place in the same chapters he says, ``The saying is old that
truth should not be spoken at all times; and those whom a sick
conscience worries into habitual violation of the maxim are imbeciles
and nuisances.''

That is strong language, but true.
None of us could \textit{live} with an habitual truth-teller;
but thank goodness none of us has to. An habitual truth-teller is simply
an impossible creature who does not exist, and never has existed.
Of course there are people who \textit{think} they
never lie, but it is not so; and this ignorance is one of the very
things that shame our so-called civilization. Everybody lies, every day;
every hour; awake; asleep; in dreams; in joy; in mourning; in silence.
Hands, feet, eyes, attitude---all convey deception.

We are liars, every one. Our mere \textit{howdy-do}
is a lie, because we do not care how you did.
To the ordinary inquirer you lie in return; for you make
no conscientious diagnostic of your case, but answer at random, and
usually miss it considerably. If a stranger calls and interrupts
you, you say with your hearty tongue, ``I'm glad to see you,'' and say
with your heartier soul, ``I wish you were with the cannibals and it was
dinner-time.'' But you did no harm, for you did not
deceive anybody nor inflict any hurt, whereas the truth would have made
you both unhappy.

I think that all this courteous lying is a sweet and loving art, and
should be cultivated. The highest perfection of politeness is
a beautiful edifice, built, from the base to the dome, of graceful and
gilded forms of charitable and unselfish lying.

What I bemoan is the growing prevalence of the brutal truth. Let us do
what we can to eradicate it. An injurious truth has no merit over an
injurious lie. Neither should ever be uttered. Whoever speaks an
injurious truth in fear of damnation for lying, should
reflect that that sort of a soul is not strictly worth saving.
Whoever tells a lie to help poor devils out of trouble,
is one of whom the angels doubtless say, ``Lo, let us exalt this
magnanimous liar.''

An injurious lie is an uncommendable thing; and so, and in the same
degree, is an injurious truth. Lying is universal: we \textit{all} do it.
Therefore, the wise thing is for us
diligently to train ourselves to lie thoughtfully, judiciously; to lie
with a good object, and not an evil one; to lie for others' advantage,
and not our own; to lie healingly, charitably, humanely, not cruelly,
hurtfully, maliciously; to lie gracefully and graciously, not awkwardly
and clumsily; to lie firmly, frankly, squarely, with head erect, not
haltingly, tortuously, with pusillanimous mien, as being ashamed of our
high calling. Then shall we be rid of the rank and pestilent truth that
is rotting the land; then shall we be great and good and beautiful, and
worthy dwellers in a world where even benign Nature habitually lies,
except when promising execrable weather.

Joking aside, I think there is much need of wise examination into what
sorts of lies are best and wholesomest to be indulged, seeing we \textit{must}
all lie and we \textit{do} all lie.


\end{document}
