% !TeX TS-program = lualatex
% !TeX encoding = UTF-8
\documentclass[../../demo.tex]{novelette-subdoc}
\begin{document}

\begin{pageiii} % Page iii. Title Page.
\null\null\null\null\null % 5 blank lines.
\format[scale=3,font=light]{Excerpts} % Each line is independent,
\format[scale=2,font=light]{From} % so you can use different scale or alignment
\format[scale=3,font=light]{Mark Twain} % for each line.
\null\null\null % 3 blank lines.
\format[scale=3]{\textit{Samuel Clemens}}
\vfill % Pushes surrounding content to top and bottom of page.
\format{Demonstration of the Novelette Document Class}
\end{pageiii}

\begin{pageiv} % Page iv. Copyright Page. Pre-styled.
Excerpts From Mark Twain\\/ Samuel Clemens\par
Mark Twain is the pseudonym of American\\
author Samuel Clemens (1835--1910)\par
The text and image are in the Public Domain\\
of the United States of America, and of nations\\
subscribing to the Berne Copyright Convention,\\
due to passage of time (over 100 years) since\\
initial publication and the death of its author.\par
This is largely a work of humorous satire.\par
Novelette project page:\\https://github.com/rallg/novelette\par
ISBN \lnum{0-00-000000000-0}\par
\end{pageiv}

% In this case, pagev is an Editor's Note. It has been styled so that
% it does not resemble a chapter.
\begin{pagev} % Page v.
\null\null\null\null\null % blank lines.
\format[font=light,scale=1.3]{Editor's Note}
\null\null\null\null\null % blank lines.
\begin{blockindent}[4,4] % Better several short lines, than few long ones.
\noindent This is a demonstration of Novelette document
class, for the LuaLaTeX typesetting engine.
The words of Samuel Clemens have been edited for appearance and flow.
Do not use this as literary reference.\par
\end{blockindent}
\end{pagev}

% Blank page vi will be inserted automatically.

\end{document}
