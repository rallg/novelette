% !TeX TS-program = lualatex
% !TeX encoding = UTF-8
\documentclass[../demo.tex]{novelettesubdoc}
\begin{document}

\begin{fullpage} % Title Page.
\null\null\null\null\null % 5 blank lines.
\formatb{EXCERPTS} % Each line is independent,
\formatc{from} % so you can use different scale or alignment
\formatd{MARK TWAIN} % for each line.
\null\null\null % 3 blank lines.
\formate{Samuel Clemens}
\vfill % Pushes surrounding content to top and bottom of page.
\formatf{Demonstration of the Novelette Document Class}
\end{fullpage}

\begin{copyrightpage}
\vfill % Pushes following content to bottom of page.
% Centered text is one way to do this, particularly for books that
% do not have a library catalog number or Catalog-in-Publication data.
% A common format is title : subtitle / author.
% Your book needs a real copyright notice. Be sure to have it!
EXCEPTS from MARK TWAIN\br/ Samuel Clemens\par
Mark Twain is the pseudonym of American\br
author Samuel Clemens (1835--1910)\par
The text and image are in the Public Domain\br
of the United States of America, and of nations\br
subscribing to the Berne Copyright Convenion,\br
due to passage of time (over 100 years) since\br
initial publication and the death of its author.\par
This is largely a work of humorous satire.\par
Novelette project page:\br https://github.com/rallg/novelette\par
ISBN 0-00-000000000-0\par
\end{copyrightpage}


% The fullpage environment sets parindent to 0. For this particular case,
% indentation is desired. \forceindent to the rescue.
\begin{fullpage} % Visual Separator.
\begin{upperpage}
\null\null\null\null % 4 blank lines.
\name{Editor's Note}
\null\null % 2 blank lines.
\end{upperpage}
\begin{blockindent}[2,2]
This is a demonstration of the Novelette document
class, compiled with the LuaLaTeX typesetting engine.
The words of Samuel Clemens have been edited for flow and appearance.
None of this should be used as literary reference.\par
\forceindent Novelette is intended for original popular fiction, rather than
non-fiction or reprint of older works. Thus, this demo is outside the intended
genres. Why \ital{original} popular fiction? Then, when paragraphs are ugly,
or page breaks are awkward, you may simply re-write your own work.
But if you must duplicate the words of someone else, then there will be places
with bad typesetting, and Novelette cannot help you.\par 
\forceindent The Novelette documentation includes a ``novelette-demo'' folder.
File ``demo.tex'' produces this \medcap{PDF}.\par
\end{blockindent}
\end{fullpage}


\blankpage % Page vi.


\end{document}
