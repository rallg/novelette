% !TeX TS-program = lualatex
% !TeX encoding = UTF-8
\documentclass[../../../demo.tex]{novelettesubdoc}
\mode{typo} % Just when this file is compiled alone.
\begin{document}

Considering the Missouri its main branch, it is the longest river in the
world---four thousand three hundred miles. It seems safe to say that it is
also the crookedest river in the world, since in one part of its journey
it uses up one thousand three hundred miles to cover the same ground that the
crow would fly over in six hundred and seventy-five.

It discharges three times as much water as the St. Lawrence, twenty-five times
as much as the Rhine, and three hundred and thirty-eight times as much as the
Thames. No other river has so vast a drainage-basin. The
Mississippi receives and carries to the Gulf water from
fifty-four subordinate rivers that are navigable by steamboats,
and from some hundreds that are navigable by flats and keels.
The area of its drainage-basin is as great as the combined areas of England,
Wales, Scotland, Ireland, France, Spain, Portugal, Germany, Austria, Italy,
and Turkey; and almost all this wide region is fertile.

It is a remarkable river in this: that instead of widening toward its
mouth, it grows narrower; grows narrower and deeper. From the junction
of the Ohio to a point half way down to the sea, the width averages a
mile in high water: thence to the sea the width steadily diminishes,
until, at the `Passes,' above the mouth, it is but little over half
a mile. At the junction of the Ohio the Mississippi's depth is
eighty-seven feet; the depth increases gradually, reaching one hundred
and twenty-nine just above the mouth.

The difference in rise and fall is also remarkable---not in the upper,
but in the lower river. The rise is tolerably uniform down to Natchez
(three hundred and sixty miles above the mouth)---about fifty feet.
But at Bayou La Fourche the river rises only twenty-four feet; at New
Orleans only fifteen, and just above the mouth only two and one half.

An article in the New Orleans \ital{Times-Democrat,}  based upon reports of
able engineers, states that the river annually empties four hundred and
six million tons of mud into the Gulf of Mexico---which brings to mind
Captain Marryat's rude name for the Mississippi---`the Great Sewer.' This
mud, solidified, would make a mass a mile square and two hundred and
forty-one feet high.

The mud deposit gradually extends the land---but only gradually; it has
extended it not quite a third of a mile in the two hundred years which
have elapsed since the river took its place in history. The belief of
the scientific people is, that the mouth used to be at Baton Rouge,
where the hills cease, and that the two hundred miles of land between
there and the Gulf was built by the river. This gives us the age of that
piece of country, without any trouble at all---one hundred and twenty
thousand years.

The Mississippi is remarkable in still another way---its disposition to
make prodigious jumps by cutting through narrow necks of land, and thus
straightening and shortening itself. More than once it has shortened
itself thirty miles at a single jump! These cut-offs have had curious
effects: they have thrown several river towns out into the rural
districts, and built up sand bars and forests in front of them. The town
of Delta used to be three miles below Vicksburg: a recent cutoff has
radically changed the position, and Delta is now \ital{two miles above}
Vicksburg. Both of these river towns have been retired to the country by that
cut-off.

The Mississippi does not alter its locality by cut-offs alone: it
is always changing its habitat \ital{bodily}---is always
moving bodily \ital{sidewise}. At Hard Times, La., the river is two miles west
of the region it used to occupy. As a result, the original site of that
settlement is not now in Louisiana at all, but on the other side of
the river, in the State of Mississippi. \ital{Nearly the whole of that one
thousand three hundred miles of old Mississippi River which La Salle
floated down in his canoes, two hundred years ago, is good solid dry
ground now}. The river lies to the right of it, in places, and to the
left of it in other places.

Although the Mississippi's mud builds land but slowly, down at the
mouth, where the Gulfs billows interfere with its work, it builds fast
enough in better protected regions higher up: for instance, Prophet's
Island contained one thousand five hundred acres of land thirty years
ago; since then the river has added seven hundred acres to it.

\end{document}
