% !TeX TS-program = lualatex
% !TeX encoding = UTF-8
\documentclass{novelette-cover}
\title{Cover: EXCERPTS from MARK TWAIN} % Synced to version 0.31.
\author{Samuel Clemens}
\coverimage{CHANGE THIS} % to new image filename after processing with cmyk4nvt.
\covertrim{w=11.35in,h=8.5in} % front, spine, rear, after trimming bleed.
\trimsize{w=5.5in,h=8.5in} % Must be same as book interior. 
\begin{document}\end{document} % One line, no spaces.

Anything written here, after document ends, will be ignored.

INSTRUCTIONS:

In this folder is image "riverboat.jp2". It is sized for the demo,
except that this image assumes a page count of 140, rather than the smaller
number of pages in the demo. So its spine width is 0.35in, according to the
standards of some commonly-used manufacturers.

Process "riverboat.jp2" using script "cmyk4nvt". That will create a new image
named "riverboat-nvtcNNNNNN.jpg" where the NNNNNN are digits.
This procedure takes awhile.

On this page, change the \coverimage setting, to match that new filename,
with its jpg extension.

Then process this file with lualatex. It completes quickly.

If you view the image "riverboat-nvtcNNNNNN.jpg" in an image viewer, its color
rendition will be wrong. It will appear too brightly colored. The same happens
if you view the finished PDF in most PDF readers. The reason is that the
software does not use the correct color transform, when converting CMYK
(the image and the PDF) to RGB (for view on screen). The "softproof" image
will show correct color rendition. In this case, the softproof will look almost
exactly like the original, unprocessed image. But your own cover image may
show more noticeable color shift, depending on its design.

The PDF has no text, so it has no fonts.

Note that riverboat.jp2 was created in "jpeg2000" format, with very low quality,
for small file size. You do not need to use this format when creating the
original image, but it looks better at low quality than "jpg" format.
Actual cover artwork should be png or high-quality jpg.


