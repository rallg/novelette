% !TeX TS-program = lualatex
% !TeX encoding = UTF-8
\documentclass[../interior-demo.tex]{subfiles}
\begin{document}

\begin{upperpage}
\name{2}
\subname{Cooper's Literary Offences}
\footnote{Excerpt from an essay, 1895. James Fenimore Cooper was a popular
novelist of the American pioneer era.}
\end{upperpage}
% The initial T is enlarged, and creates a visible gap between itself and the
% following smallcap h. This can be fixed by kerning. The kern must be placed
% within the braces, immediately following T. You may use the \kern command,
% but Novelette also provides \sk{number} where number is the multiple
% (positive or negative) of vertical stem widths, at curent scale.
%\init{T\sk{-1.3}}[here are rules]
There are rules governing literary art in romantic fiction.

% Novelette does not use lists. Although it is possible to change the
% numerals to lining, using \lnum{numeral}, they look better as oldstyle.
1. They require that a tale shall accomplish something and arrive
somewhere.

2. They require that the episodes of a tale shall be necessary parts of
the tale, and shall help to develop it.

3. They require that the personages in a tale shall be alive, except in
the case of corpses, and that always the reader shall be able to tell
the corpses from the others.

4. They require that the personages in a tale, both dead and alive,
shall exhibit a sufficient excuse for being there.

5. They require that when the personages of a tale deal in conversation,
the talk shall sound like human talk, and be talk such as human
beings would be likely to talk in the given circumstances, and have
a discoverable meaning, also a discoverable purpose, and a show of
relevancy, and remain in the neighborhood of the subject in hand, and
be interesting to the reader, and help out the tale, and stop when the
people cannot think of anything more to say.

6. They require that when the author describes the character of a
personage in his tale, the conduct and conversation of that personage
shall justify said description.

7. They require that when a personage talks like a moneyed college
graduate in the beginning of a paragraph, he shall not talk like an uneducated
workman in the end of it.

8. They require that crass stupidities shall not be played upon the
reader, by either the author or the people in the tale.

9. They require that the personages of a tale shall confine themselves
to possibilities and let miracles alone; or, if they venture a miracle,
the author must so plausibly set it forth as to make it look possible
and reasonable.

10. They require that the author shall make the reader feel a deep
interest in the personages of his tale and in their fate.

11. They require that the characters in a tale shall be so clearly
defined that the reader can tell beforehand what each will do in a given
emergency.

Cooper's gift in the way of invention was not a rich endowment; but
such as it was he liked to work it, he was pleased with the effects,
and indeed he did some quite sweet things with it. In his little box of
stage properties he kept six or eight cunning devices, tricks, artifices
for his savages and woodsmen to deceive and circumvent each other with,
and he was never so happy as when he was working these innocent things.

A favorite one was to make a moccasined person tread
in the tracks of the moccasined enemy, and thus hide his own trail.
Cooper wore out barrels and barrels of moccasins in working that trick.

Another stage-property that he pulled out of his box pretty frequently
was his broken twig. He prized his broken twig above all the rest of his
effects, and worked it the hardest. It is a restful chapter in any book
of his when somebody doesn't step on a dry twig. Every time a Cooper person is
in peril, and absolute silence is worth four dollars a minute, he is
sure to step on a dry twig. There may be a hundred handier things to
step on, but that wouldn't satisfy Cooper, who requires him to turn
out and find a dry twig.

Cooper was a sailor---a naval officer; yet he gravely tells us how
a vessel, driving towards a lee shore in a gale, is steered for a
particular spot by her skipper because he knows of an undertow there
which will hold her back against the gale and save her. For just pure
woodcraft, or sailorcraft, or whatever it is, isn't that neat?

For several years Cooper was daily in the society of artillery, and he ought
to have noticed that when a cannon-ball strikes the ground it either
buries itself or skips a hundred feet or so; skips again a hundred feet
or so---and so on, till finally it gets tired and rolls. Now in one place
he loses some characters in the edge of a wood
near a plain at night in a fog. They hear a cannonblast, and a
cannon-ball presently comes rolling into the wood and stops at their
feet. The heros strike out promptly and follows the track of that cannon-ball
across the plain through the dense fog and finds the fort.

If Cooper had any real knowledge of Nature's ways of doing
things, he had a most delicate art in concealing the fact. For instance:
one of his experts has lost the trail of a person he is tracking through the
forest. Apparently that trail is hopelessly lost. The expert was not stumped
for long. He turned a running
stream out of its course, and there, in the slush in its old bed, were
that person's moccasin-tracks. The current did not wash them away, as
it would have done in all other like cases--no, even the eternal laws
of Nature have to vacate when Cooper wants to put up a delicate job of
woodcraft on the reader.

Cooper hadn't any more invention than a horse; and I don't mean a
high-class horse, either; I mean a clothes-horse. It would be very
difficult to find a really clever ``situation'' in Cooper's books, and
still more difficult to find one of any kind which he has failed to
render absurd by his handling of it.

Cooper's proudest creations in the way of ``situations'' suffer
from the absence of the observer's gift. His eye was
splendidly inaccurate. Cooper seldom saw anything correctly. He saw
nearly all things as through a glass eye, darkly. Of course a man who
cannot see the commonest little every-day matters accurately is
working at a disadvantage when he is constructing a ``situation.''

The conversations in the Cooper books have a curious sound in our
ears. To believe that such talk really ever came out of people's mouths
would be to believe that time was of no value to
a person who thought he had something to say; when it was the custom
to spread a two-minute remark out to ten; when a man's mouth was a
rolling-mill, and busied itself all day long in turning four-foot pigs
of thought into thirty-foot bars of conversational railroad iron by
attenuation; when subjects were seldom faithfully stuck to, but the talk
wandered all around and arrived nowhere; when conversations consisted
mainly of irrelevancies, with here and there a relevancy, a relevancy
with an embarrassed look, as not being able to explain how it got there.

Cooper was certainly not a master in the construction of dialogue.
Inaccurate observation defeated him here as it defeated him in so many
other enterprises of his. He even failed to notice that the man who
talks corrupt English six days in the week must and will talk it on
the seventh, and can't help himself.

Cooper's word-sense was singularly dull. When a person has a poor ear
for music he will flat and sharp right along without knowing it. He
keeps near the tune, but it is not the tune. When a person has a poor
ear for words, the result is a literary flatting and sharping; you
perceive what he is intending to say, but you also perceive that he
doesn't say it. This is Cooper. He was not a word-musician. His ear was
satisfied with the approximate word.

\end{document}
