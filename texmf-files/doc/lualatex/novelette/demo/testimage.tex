% !TeX TS-program = lualatex
% !TeX encoding = UTF-8
\documentclass{novelette} % Synced with v0.12.
\mode{final}
\title{Test Image}
\author{Novelette}
\booksize{w=5.5in,h=8.5in}
\begin{document}

This shows how Novelette handles an interior image. The processing
uses \mono{bash} command line, with ImageMagick installed.

1. Copy this novelette-demo directory to your home directory, where you have
read/write privileges. Do not attempt to use it where it is installed.

2. Also copy directory novelette-scripts to your home directory.

3. Copy image testimage.png into novelette-scripts. Open a Terminal there,
and process the image:

\mono{bash mono4nvt testimage.png}

4. That produces a new image, testimage-nvtmN.png, where N is several digits.
The exact N may depend on your version of ImageMagick. Transfer this image
into novelette-demo directory.

5. Compile this file: \mono{lualatex testimage}

6. The resulting PDF is in draft mode. You will see the original testimage.png.

\memo{\image{testimage.png}}
\image{novelette-nvtm134591.png}

7. Near the top of this file, change the mode from draft to final.
Compile this file again. This time, the compiler will throw an error.
That is because the original testimage was not processed by script.

8. Now edit this file, and change the image filename
to whatever filename was produced by the script. Compile again, in final
mode. This time, the image is accepted. The PDF conforms to PDF/X-1a:2001
with default SWOP Output Intent.

\makeatletter\nocle@rtoendtrue\makeatother
\end{document}
