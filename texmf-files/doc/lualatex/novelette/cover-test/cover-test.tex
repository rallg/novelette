% !TeX TS-program = lualatex
% !TeX encoding = UTF-8
\documentclass{novelette} % Add [cover] option, when ready.
\title{Sample Cover} % Synced to version 0.30.
\author{Novelette}
\coverimage{CHANGE-THIS}
\covertrim{w=11.464in,h=8.5in} % Specific to the booksize and spine width.
\trimsize{w=5.5in,h=8.5in}
\begin{document}

This is an example of how to produce a \medcap{PDF/X-1a:2001} conformant file,
from an ordinary image of the correct cover size. These instructions are
for a computer with the BASH command line, and GraphicsMagick installed.

1. Copy the cover-test directory to your home directory, where you
have read/write privileges. Do not attempt to process these files where
they are originally installed, because you probably cannot write files there.

2. Copy the novelette-scripts directory to your home directory.

3. Copy file testcover.jpg into novelette-scripts.
Launch a Terminal there, and process the image. Command:

bash cmyk4nvt testcover.jpg

4. That takes about a minute. When it is done, you will have two new images
there. One of them is named testcover-softproof.jpg. You may compare it
to the original testcover.jpg in any image viewer. Notice how the colors
are slightly dull, and the dark areas are less dark, in the softproof.
This emulates how the image will look as a printed cover.

The other image is testcover-nvtcN.jpg, where N is a multi-digit number.
The number may depend on which version of ImageMagick you used. This is
in CMYK color space. Image viewers will not show you accurate colors.
That is because their method for viewing CMYK uses a generic transform that
is incorect for this type of image.

5. Transfer the testcover-nvtcN.jpg image into the cover-test
directory. You may keep or discard the softproof.

5. At the top of this cover-test.tex file, add the [cover]
document class option, and change the coverimage filename to whatever was
produced by the script. It is important to use that exact filename, with its
jpg extension.

6. Then process this tex file:

lualatex cover-test

\null

7. The resulting PDF is the cover image, \lnum{PDF/X-1a:2001} with SWOP Output
Intent. You may view it in a PDF reader, but its colors will appear wrong.
This is because the PDF reader is not emulating a printer.

A few PDF readers (possibly Adobe Reader) may be able to diagnose the PDF,
and apply a suitable color transform.
But do not rely on what you see. Only the softproof shows the transform.

8. None of this text appears in the cover image PDF. No fonts are used.
When Novelette processes with [cover] option, it ignores any Preamble
settings unrelated to the cover, and ignores anything in the document body.

\makeatletter\nocle@rtoendtrue\makeatother
\end{document}
