% !TeX TS-program = lualatex
% !TeX encoding = UTF-8
\documentclass{novelette}
\title{Warn Me, Bro}
\author{Da Bro}
\booksize{w=5.5in,h=8.5in}
\mode{final}
%\margins{t=.5in,o=.5in,b=.5in,i=.5in}
%\warnalert{on}
\begin{document}

This original text of this document does not issue any Errors, Warnings,
or ALERTs. Edit it, and see what happens.

1. Usually, print services expect the inside margin to be larger than the
outside margin, to allow for glue at the spine. Novelette does that, as
default. But if you uncomment the above margins command, it sets the side
margins to the same value. This will write an ALERT in the log summary.
Maybe that is what you meant to do, maybe not.

2. Final mode requires both title and author. If you comment-out either
of them, then Novelette will throw a Warning, and the PDF will not be
in final mode. The Warning will no be ussed if you also comment-out
the mode (default is draft).

3. The booksize is mandatory. If you comnment-out booksize, then a fatal error
is thrown, and the compiler will not continue.

4. An ALERT is written to the log summary, if the document text uses the
un-escaped percent symbol anywhere. Did you mean that as a comment, or did you
intend to print the percent symbol? To see the ALERT, type the percent symbol
somewhere here.

5. An ALERT is written to the log summary, if the document text uses the
straight double quotes anywhere. This is because TeX and word processors
behave differently, when they encounter straight double quotes. To see the
ALERT, type straight dobule quotes somewhere here.

6. Normally, an ALERT does not do anything other than write a message to
the log summary. If ALERTS are important to you, then un-comment
the above warnalert command. Then, any ALERT will also throw a Warning.
The Warning will be noticed by GUI programs. Also, the Warning will
prevent final mode.

\end{document}
% Anything after \end{document} is not processed. So the percent symbol,
% and the " straight double quotes, do not throw ALERTs here.
